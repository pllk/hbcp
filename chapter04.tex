\chapter{Estruturas de Dados}

\index{data structure}

Uma \key{estrutura de dados} é uma forma de armazenar
dados na memória de um computador.
É importante escolher uma estrutura de dados apropriada para um problema,
porque cada estrutura de dados tem suas próprias
vantagens e desvantagens.
A questão crucial é: quais operações
são eficientes na estrutura de dados escolhida?

Este capítulo apresenta as estruturas de dados mais importantes
na biblioteca padrão do C++.
É uma boa ideia usar a biblioteca padrão
sempre que possível,
porque isso economizará muito tempo.
Mais adiante no livro, aprenderemos sobre mais sofisticadas
estruturas de dados que não estão disponíveis
na biblioteca padrão.

\section{Vetores Dinâmicos}

\index{dynamic array}
\index{vector}

Um \key{vetor dinâmico} é um vetor cujo
tamanho pode ser alterado durante a execução
do programa.
O vetor dinâmico mais popular em C++ é
a estrutura \texttt{vector},
que pode ser usada quase como um vetor comum.

O código a seguir cria um vetor vazio e
adiciona três elementos a ele:

\begin{lstlisting}
vector<int> v;
v.push_back(3); // [3]
v.push_back(2); // [3,2]
v.push_back(5); // [3,2,5]
\end{lstlisting}

Depois disso, os elementos podem ser acessados como em um vetor comum:

\begin{lstlisting}
cout << v[0] << "\n"; // 3
cout << v[1] << "\n"; // 2
cout << v[2] << "\n"; // 5
\end{lstlisting}

A função \texttt{size} retorna o número de elementos no vetor.
O código a seguir itera através
do vetor e imprime todos os elementos nele:

\begin{lstlisting}
for (int i = 0; i < v.size(); i++) {
    cout << v[i] << "\n";
}
\end{lstlisting}

\begin{samepage}
Uma maneira mais curta de iterar através de um vetor é a seguinte:

\begin{lstlisting}
for (auto x : v) {
    cout << x << "\n";
}
\end{lstlisting}
\end{samepage}

A função \texttt{back} retorna o último elemento
no vetor, e
a função \texttt{pop\_back} remove o último elemento:

\begin{lstlisting}
vector<int> v;
v.push_back(5);
v.push_back(2);
cout << v.back() << "\n"; // 2
v.pop_back();
cout << v.back() << "\n"; // 5
\end{lstlisting}

O código a seguir cria um vetor com cinco elementos:

\begin{lstlisting}
vector<int> v = {2,4,2,5,1};
\end{lstlisting}

Outra maneira de criar um vetor é fornecer o número
de elementos e o valor inicial para cada elemento:

\begin{lstlisting}
// tamanho 10, valor inicial 0
vector<int> v(10);
\end{lstlisting}
\begin{lstlisting}
// tamanho 10, valor inicial 5
vector<int> v(10, 5);
\end{lstlisting}

A implementação interna de um vector
usa um vetor comum.
Se o tamanho do vetor aumenta e
o vetor se torna muito pequeno,
um novo vetor é alocado e todos os
elementos são movidos para o novo vetor.
No entanto, isso não acontece com frequência e a
complexidade de tempo média de
\texttt{push\_back} é $O(1)$.

\index{string}

A estrutura \texttt{string}
também é um vetor dinâmico que pode ser usado quase como um vetor.
Além disso, há uma sintaxe especial para strings
que não está disponível em outras estruturas de dados.
Strings podem ser combinadas usando o símbolo \texttt{+}.
A função $\texttt{substr}(k,x)$ retorna a substring
que começa na posição $k$ e tem comprimento $x$,
e a função $\texttt{find}(\texttt{t})$ encontra a posição
da primeira ocorrência de uma substring \texttt{t}.

O código a seguir apresenta algumas operações com strings:

\begin{lstlisting}
string a = "hatti";
string b = a+a;
cout << b << "\n"; // hattihatti
b[5] = 'v';
cout << b << "\n"; // hattivatti
string c = b.substr(3,4);
cout << c << "\n"; // tiva
\end{lstlisting}

\section{Estruturas de Conjunto}

\index{set}

Um \key{conjunto} é uma estrutura de dados que
mantém uma coleção de elementos.
As operações básicas de conjuntos são inserção de elemento,
pesquisa e remoção.

A biblioteca padrão do C++ contém duas implementações de conjunto:
A estrutura \texttt{set} é baseada em uma árvore binária balanceada e suas operações funcionam em tempo $O(\log n)$.
A estrutura \texttt{unordered\_set} usa hashing,
e suas operações funcionam em tempo $O(1)$ em média.

A escolha de qual implementação de conjunto usar
é frequentemente uma questão de gosto.
O benefício da estrutura \texttt{set}
é que ela mantém a ordem dos elementos
e fornece funções que não estão disponíveis
em \texttt{unordered\_set}.
Por outro lado, \texttt{unordered\_set}
pode ser mais eficiente.

O código a seguir cria um conjunto
que contém inteiros,
e mostra algumas das operações.
A função \texttt{insert} adiciona um elemento ao conjunto,
a função \texttt{count} retorna o número de ocorrências
de um elemento no conjunto,
e a função \texttt{erase} remove um elemento do conjunto.

\begin{lstlisting}
set<int> s;
s.insert(3);
s.insert(2);
s.insert(5);
cout << s.count(3) << "\n"; // 1
cout << s.count(4) << "\n"; // 0
s.erase(3);
s.insert(4);
cout << s.count(3) << "\n"; // 0
cout << s.count(4) << "\n"; // 1
\end{lstlisting}

Um conjunto pode ser usado principalmente como um vetor,
mas não é possível acessar
os elementos usando a notação \texttt{[]}.
O código a seguir cria um conjunto,
imprime o número de elementos nele e então
itera por todos os elementos:
\begin{lstlisting}
set<int> s = {2,5,6,8};
cout << s.size() << "\n"; // 4
for (auto x : s) {
    cout << x << "\n";
}
\end{lstlisting}

Uma propriedade importante dos conjuntos é
que todos os seus elementos são \emph{distintos}.
Assim, a função \texttt{count} sempre retorna
0 (o elemento não está no conjunto)
ou 1 (o elemento está no conjunto),
e a função \texttt{insert} nunca adiciona
um elemento ao conjunto se ele já estiver lá.
O código a seguir ilustra isso:

\begin{lstlisting}
set<int> s;
s.insert(5);
s.insert(5);
s.insert(5);
cout << s.count(5) << "\n"; // 1
\end{lstlisting}

C++ também contém as estruturas
\texttt{multiset} e \texttt{unordered\_multiset}
que, de outra forma, funcionam como \texttt{set}
e \texttt{unordered\_set}
mas podem conter várias instâncias de um elemento.
Por exemplo, no código a seguir, todas as três instâncias
do número 5 são adicionadas a um multiconjunto:

\begin{lstlisting}
multiset<int> s;
s.insert(5);
s.insert(5);
s.insert(5);
cout << s.count(5) << "\n"; // 3
\end{lstlisting}
A função \texttt{erase} remove
todas as instâncias de um elemento
de um multiconjunto:
\begin{lstlisting}
s.erase(5);
cout << s.count(5) << "\n"; // 0
\end{lstlisting}
Frequentemente, apenas uma instância deve ser removida,
o que pode ser feito da seguinte forma:
\begin{lstlisting}
s.erase(s.find(5));
cout << s.count(5) << "\n"; // 2
\end{lstlisting}

\section{Estruturas de Mapa}

\index{map}

Um \key{mapa} é um vetor generalizado
que consiste em pares chave-valor.
Enquanto as chaves em um vetor comum são sempre
os inteiros consecutivos $0,1,\ldots,n-1$,
onde $n$ é o tamanho do vetor,
as chaves em um mapa podem ser de qualquer tipo de dados e
não precisam ser valores consecutivos.

A biblioteca padrão do C++ contém duas implementações de mapa
que correspondem às implementações de conjunto: a estrutura
\texttt{map} é baseada em uma árvore binária balanceada e acessar elementos
leva tempo $O(\log n)$,
enquanto a estrutura
\texttt{unordered\_map} usa hashing
e acessar elementos leva tempo $O(1)$ em média.

O código a seguir cria um mapa
onde as chaves são strings e os valores são inteiros:

\begin{lstlisting}
map<string,int> m;
m["monkey"] = 4;
m["banana"] = 3;
m["harpsichord"] = 9;
cout << m["banana"] << "\n"; // 3
\end{lstlisting}

Se o valor de uma chave for solicitado
mas o mapa não o contém,
a chave é adicionada automaticamente ao mapa com
um valor padrão.
Por exemplo, no código a seguir,
a chave ''aybabtu'' com valor 0
é adicionada ao mapa.

\begin{lstlisting}
map<string,int> m;
cout << m["aybabtu"] << "\n"; // 0
\end{lstlisting}
A função \texttt{count} verifica
se uma chave existe em um mapa:
\begin{lstlisting}
if (m.count("aybabtu")) {
    // a chave existe
}
\end{lstlisting}
O código a seguir imprime todas as chaves e valores
em um mapa:
\begin{lstlisting}
for (auto x : m) {
    cout << x.first << " " << x.second << "\n";
}
\end{lstlisting}

\section{Iteradores e Intervalos}

\index{iterator}

Muitas funções na biblioteca padrão do C++
operam com iteradores.
Um \key{iterador} é uma variável que aponta
para um elemento em uma estrutura de dados.

Os iteradores frequentemente usados \texttt{begin}
e \texttt{end} definem um intervalo que contém
todos os elementos em uma estrutura de dados.
O iterador \texttt{begin} aponta para
o primeiro elemento na estrutura de dados,
e o iterador \texttt{end} aponta para
a posição \emph{após} o último elemento.
A situação é a seguinte:

\begin{center}
\begin{tabular}{llllllllll}
\{ & 3, & 4, & 6, & 8, & 12, & 13, & 14, & 17 & \} \\
& $\uparrow$ & & & & & & & & $\uparrow$ \\
& \multicolumn{3}{l}{\texttt{s.begin()}} & & & & & & \texttt{s.end()} \\
\end{tabular}
\end{center}

Observe a assimetria nos iteradores:
\texttt{s.begin()} aponta para um elemento na estrutura de dados,
enquanto \texttt{s.end()} aponta para fora da estrutura de dados.
Assim, o intervalo definido pelos iteradores é \emph{semiaberto}.

\subsubsection{Trabalhando com Intervalos}

Iteradores são usados em funções da biblioteca padrão do C++
que recebem um intervalo de elementos em uma estrutura de dados.
Normalmente, queremos processar todos os elementos em uma
estrutura de dados, então os iteradores
\texttt{begin} e \texttt{end} são fornecidos para a função.

Por exemplo, o código a seguir ordena um vetor
usando a função \texttt{sort},
então inverte a ordem dos elementos usando a função
\texttt{reverse}, e finalmente embaralha a ordem de
os elementos usando a função \texttt{random\_shuffle}.

\index{sort@\texttt{sort}}
\index{reverse@\texttt{reverse}}
\index{random\_shuffle@\texttt{random\_shuffle}}

\begin{lstlisting}
sort(v.begin(), v.end());
reverse(v.begin(), v.end());
random_shuffle(v.begin(), v.end());
\end{lstlisting}

Essas funções também podem ser usadas com um vetor comum.
Nesse caso, as funções recebem ponteiros para o vetor
em vez de iteradores:

\newpage
\begin{lstlisting}
sort(a, a+n);
reverse(a, a+n);
random_shuffle(a, a+n);
\end{lstlisting}

\subsubsection{Iteradores de Conjunto}

Iteradores são frequentemente usados para acessar
elementos de um conjunto.
O código a seguir cria um iterador
\texttt{it} que aponta para o menor elemento em um conjunto:
\begin{lstlisting}
set<int>::iterator it = s.begin();
\end{lstlisting}
Uma maneira mais curta de escrever o código é a seguinte:
\begin{lstlisting}
auto it = s.begin();
\end{lstlisting}
O elemento para o qual um iterador aponta
pode ser acessado usando o símbolo \texttt{*}.
Por exemplo, o código a seguir imprime
o primeiro elemento no conjunto:

\begin{lstlisting}
auto it = s.begin();
cout << *it << "\n";
\end{lstlisting}

Iteradores podem ser movidos usando os operadores
\texttt{++} (para frente) e \texttt{--} (para trás),
o que significa que o iterador se move para o próximo
ou anterior elemento no conjunto.

O código a seguir imprime todos os elementos
em ordem crescente:
\begin{lstlisting}
for (auto it = s.begin(); it != s.end(); it++) {
    cout << *it << "\n";
}
\end{lstlisting}
O código a seguir imprime o maior elemento no conjunto:
\begin{lstlisting}
auto it = s.end(); it--;
cout << *it << "\n";
\end{lstlisting}

A função $\texttt{find}(x)$ retorna um iterador
que aponta para um elemento cujo valor é $x$.
No entanto, se o conjunto não contém $x$,
o iterador será \texttt{end}.

\begin{lstlisting}
auto it = s.find(x);
if (it == s.end()) {
    // x nao foi encontrado
}
\end{lstlisting}

A função $\texttt{lower\_bound}(x)$ retorna
um iterador para o menor elemento no conjunto
cujo valor é \emph{pelo menos} $x$, e
a função $\texttt{upper\_bound}(x)$
retorna um iterador para o menor elemento no conjunto
cujo valor é \emph{maior que} $x$.
Em ambas as funções, se tal elemento não existe,
o valor de retorno é \texttt{end}.
Essas funções não são suportadas pela
estrutura \texttt{unordered\_set} que
não mantém a ordem dos elementos.

\begin{samepage}
Por exemplo, o código a seguir encontra o elemento
mais próximo a $x$:

\begin{lstlisting}
auto it = s.lower_bound(x);
if (it == s.begin()) {
    cout << *it << "\n";
} else if (it == s.end()) {
    it--;
    cout << *it << "\n";
} else {
    int a = *it; it--;
    int b = *it;
    if (x-b < a-x) cout << b << "\n";
    else cout << a << "\n";
}
\end{lstlisting}

O código assume que o conjunto não está vazio,
e passa por todos os casos possíveis
usando um iterador \texttt{it}.
Primeiro, o iterador aponta para o menor
elemento cujo valor é pelo menos $x$.
Se \texttt{it} for igual a \texttt{begin},
o elemento correspondente está mais próximo de $x$.
Se \texttt{it} for igual a \texttt{end},
o maior elemento no conjunto está mais próximo de $x$.
Se nenhum dos casos anteriores for válido,
o elemento mais próximo a $x$ é o
elemento que corresponde a \texttt{it} ou o elemento anterior.
\end{samepage}

\section{Outras Estruturas}

\subsubsection{Bitset}

\index{bitset}

Um \key{bitset} é um vetor
cujo cada valor é 0 ou 1.
Por exemplo, o código a seguir cria
um bitset que contém 10 elementos:
\begin{lstlisting}
bitset<10> s;
s[1] = 1;
s[3] = 1;
s[4] = 1;
s[7] = 1;
cout << s[4] << "\n"; // 1
cout << s[5] << "\n"; // 0
\end{lstlisting}

O benefício de usar bitsets é que
eles requerem menos memória do que vetores comuns,
porque cada elemento em um bitset apenas
usa um bit de memória.
Por exemplo, 
se $n$ bits são armazenados em um vetor \texttt{int},
$32n$ bits de memória serão usados,
mas um bitset correspondente requer apenas $n$ bits de memória.
Além disso, os valores de um bitset
podem ser manipulados eficientemente usando
operadores de bits, o que torna possível
otimizar algoritmos usando conjuntos de bits.

O código a seguir mostra outra maneira de criar o bitset acima:
\begin{lstlisting}
bitset<10> s(string("0010011010")); // da direita para a esquerda
cout << s[4] << "\n"; // 1
cout << s[5] << "\n"; // 0
\end{lstlisting}

A função \texttt{count} retorna o número
de uns no bitset:

\begin{lstlisting}
bitset<10> s(string("0010011010"));
cout << s.count() << "\n"; // 4
\end{lstlisting}

O código a seguir mostra exemplos de uso de operações de bits:
\begin{lstlisting}
bitset<10> a(string("0010110110"));
bitset<10> b(string("1011011000"));
cout << (a&b) << "\n"; // 0010010000
cout << (a|b) << "\n"; // 1011111110
cout << (a^b) << "\n"; // 1001101110
\end{lstlisting}

\subsubsection{Deque}

\index{deque}

Um \key{deque} é um vetor dinâmico
cujo tamanho pode ser eficientemente
alterado em ambas as extremidades do vetor.
Como um vetor, um deque fornece as funções
\texttt{push\_back} e \texttt{pop\_back}, mas
também inclui as funções
\texttt{push\_front} e \texttt{pop\_front}
que não estão disponíveis em um vetor.

Um deque pode ser usado da seguinte forma:
\begin{lstlisting}
deque<int> d;
d.push_back(5); // [5]
d.push_back(2); // [5,2]
d.push_front(3); // [3,5,2]
d.pop_back(); // [3,5]
d.pop_front(); // [5]
\end{lstlisting}

A implementação interna de um deque
é mais complexa do que a de um vetor,
e por esta razão, um deque é mais lento que um vetor.
Ainda assim, adicionar e remover
elementos leva tempo $O(1)$ em média em ambas as extremidades.

\subsubsection{Pilha}

\index{stack}

Uma \key{pilha}
é uma estrutura de dados que fornece duas
operações de tempo $O(1)$:
adicionar um elemento ao topo,
e remover um elemento do topo.
Só é possível acessar o topo
elemento de uma pilha.

O código a seguir mostra como uma pilha pode ser usada:
\begin{lstlisting}
stack<int> s;
s.push(3);
s.push(2);
s.push(5);
cout << s.top(); // 5
s.pop();
cout << s.top(); // 2
\end{lstlisting}
\subsubsection{Fila}

\index{queue}

Uma \key{fila} também
fornece duas operações de tempo $O(1)$:
adicionar um elemento ao final da fila,
e remover o primeiro elemento da fila.
Só é possível acessar o primeiro
e último elemento de uma fila.

O código a seguir mostra como uma fila pode ser usada:
\begin{lstlisting}
queue<int> q;
q.push(3);
q.push(2);
q.push(5);
cout << q.front(); // 3
q.pop();
cout << q.front(); // 2
\end{lstlisting}

\subsubsection{Fila de Prioridade}

\index{priority queue}
\index{heap}

Uma \key{fila de prioridade}
mantém um conjunto de elementos.
As operações suportadas são inserção e,
dependendo do tipo de fila,
recuperação e remoção de
o elemento mínimo ou máximo.
A inserção e remoção levam tempo $O(\log n)$,
e a recuperação leva tempo $O(1)$.

Enquanto um conjunto ordenado suporta eficientemente
todas as operações de uma fila de prioridade,
o benefício de usar uma fila de prioridade é
que ela tem fatores constantes menores.
Uma fila de prioridade é geralmente implementada usando
uma estrutura de heap que é muito mais simples do que uma
árvore binária balanceada usada em um conjunto ordenado.

\begin{samepage}
Por padrão, os elementos em uma
fila de prioridade C++ são classificados em ordem decrescente,
e é possível encontrar e remover o
maior elemento da fila.
O código a seguir ilustra isso:

\begin{lstlisting}
priority_queue<int> q;
q.push(3);
q.push(5);
q.push(7);
q.push(2);
cout << q.top() << "\n"; // 7
q.pop();
cout << q.top() << "\n"; // 5
q.pop();
q.push(6);
cout << q.top() << "\n"; // 6
q.pop();
\end{lstlisting}
\end{samepage}

Se quisermos criar uma fila de prioridade
que suporte encontrar e remover
o menor elemento,
podemos fazê-lo da seguinte forma:

\begin{lstlisting}
priority_queue<int,vector<int>,greater<int>> q;
\end{lstlisting}

\subsubsection{Estruturas de Dados Baseadas em Políticas}

O compilador \texttt{g++} também suporta
algumas estruturas de dados que não fazem parte
da biblioteca padrão C++.
Tais estruturas são chamadas de estruturas de dados
\emph{baseadas em políticas}.
Para usar essas estruturas, as seguintes linhas
devem ser adicionadas ao código:
\begin{lstlisting}
#include <ext/pb_ds/assoc_container.hpp>
using namespace __gnu_pbds; 
\end{lstlisting}
Depois disso, podemos definir uma estrutura de dados \texttt{indexed\_set} que
é como \texttt{set} mas pode ser indexada como um vetor.
A definição para valores \texttt{int} é a seguinte:
\begin{lstlisting}
typedef tree<int,null_type,less<int>,rb_tree_tag,
             tree_order_statistics_node_update> indexed_set; 
\end{lstlisting}
Agora podemos criar um conjunto da seguinte forma:
\begin{lstlisting}
indexed_set s;
s.insert(2);
s.insert(3);
s.insert(7);
s.insert(9);
\end{lstlisting}
A especialidade deste conjunto é que temos acesso a
os índices que os elementos teriam em um vetor ordenado.
A função $\texttt{find\_by\_order}$ retorna
um iterador para o elemento em uma determinada posição:
\begin{lstlisting}
auto x = s.find_by_order(2);
cout << *x << "\n"; // 7
\end{lstlisting}
E a função $\texttt{order\_of\_key}$
retorna a posição de um determinado elemento:
\begin{lstlisting}
cout << s.order_of_key(7) << "\n"; // 2
\end{lstlisting}
Se o elemento não aparecer no conjunto,
obtemos a posição que o elemento teria
no conjunto:
\begin{lstlisting}
cout << s.order_of_key(6) << "\n"; // 2
cout << s.order_of_key(8) << "\n"; // 3
\end{lstlisting}
Ambas as funções funcionam em tempo logarítmico.

\section{Comparação com Ordenação}

Muitas vezes é possível resolver um problema
usando estruturas de dados ou ordenação.
Às vezes, existem diferenças notáveis
na eficiência real dessas abordagens,
que podem estar ocultas em suas complexidades de tempo.

Vamos considerar um problema onde
recebemos duas listas $A$ e $B$
que contêm $n$ elementos.
Nossa tarefa é calcular o número de elementos
que pertencem a ambas as listas.
Por exemplo, para as listas
\[A = [5,2,8,9] \hspace{10px} \textrm{e} \hspace{10px} B = [3,2,9,5],\]
a resposta é 3 porque os números 2, 5
e 9 pertencem a ambas as listas.

Uma solução direta para o problema é
percorrer todos os pares de elementos em tempo $O(n^2)$,
mas a seguir vamos nos concentrar em
algoritmos mais eficientes.

\subsubsection{Algoritmo 1}

Construímos um conjunto dos elementos que aparecem em $A$,
e depois disso, iteramos pelos elementos
de $B$ e verificamos para cada elemento se ele
também pertence a $A$.
Isso é eficiente porque os elementos de $A$
estão em um conjunto.
Usando a estrutura \texttt{set},
a complexidade de tempo do algoritmo é $O(n \log n)$.

\subsubsection{Algoritmo 2}

Não é necessário manter um conjunto ordenado,
então, em vez da estrutura \texttt{set}
também podemos usar a estrutura \texttt{unordered\_set}.
Esta é uma maneira fácil de tornar o algoritmo
mais eficiente, porque só temos que mudar
a estrutura de dados subjacente.
A complexidade de tempo do novo algoritmo é $O(n)$.

\subsubsection{Algoritmo 3}

Em vez de estruturas de dados, podemos usar a ordenação.
Primeiro, ordenamos as listas $A$ e $B$.
Depois disso, iteramos pelas duas listas
ao mesmo tempo e encontramos os elementos comuns.
A complexidade de tempo da ordenação é $O(n \log n)$,
e o resto do algoritmo funciona em tempo $O(n)$,
então a complexidade de tempo total é $O(n \log n)$.

\subsubsection{Comparação de Eficiência}

A tabela a seguir mostra a eficiência
dos algoritmos acima quando $n$ varia e
os elementos das listas são inteiros aleatórios
entre $1 \ldots 10^9$:

\begin{center}
\begin{tabular}{rrrr}
$n$ & Algoritmo 1 & Algoritmo 2 & Algoritmo 3 \\
\hline
$10^6$ & $1.5$ s & $0.3$ s & $0.2$ s \\
$2 \cdot 10^6$ & $3.7$ s & $0.8$ s & $0.3$ s \\
$3 \cdot 10^6$ & $5.7$ s & $1.3$ s & $0.5$ s \\
$4 \cdot 10^6$ & $7.7$ s & $1.7$ s & $0.7$ s \\
$5 \cdot 10^6$ & $10.0$ s & $2.3$ s & $0.9$ s \\
\end{tabular}
\end{center}

Os algoritmos 1 e 2 são iguais, exceto que
eles usam estruturas de conjunto diferentes.
Neste problema, esta escolha tem um efeito importante sobre
o tempo de execução, porque o Algoritmo 2
é 4 a 5 vezes mais rápido que o Algoritmo 1.

No entanto, o algoritmo mais eficiente é o Algoritmo 3
que usa ordenação.
Ele usa apenas metade do tempo em comparação com o Algoritmo 2.
Curiosamente, a complexidade de tempo do Algoritmo 1
e do Algoritmo 3 é $O(n \log n)$,
mas apesar disso, o Algoritmo 3 é dez vezes mais rápido.
Isso pode ser explicado pelo fato de que
a ordenação é um procedimento simples e é feito
apenas uma vez no início do Algoritmo 3,
e o resto do algoritmo funciona em tempo linear.
Por outro lado,
o Algoritmo 1 mantém uma árvore binária balanceada complexa
durante todo o algoritmo.
